\documentclass[a4paper,openright,12pt]{article}
\usepackage[utf8]{inputenc}
\usepackage{graphicx} 
\usepackage{subfigure}
\usepackage[mathscr]{eucal}
\usepackage{titling}
\usepackage{float}
\usepackage{amsmath}
\usepackage{afterpage}
\usepackage{vmargin}
\usepackage[spanish]{babel}
\usepackage{eurosym} 
\usepackage{multirow} 
\usepackage{cite}
\usepackage{url}

\setlength{\parskip}{1ex} 
\setpapersize{A4}	   %  DIN A4
\setmargins{3cm}    % margen izquierdo
{3.5cm}                     % margen superior
{15cm}                       % anchura del texto
{22.5cm}                   % altura del texto
{10pt}                         % altura de los encabezados
{1cm}                         % espacio entre el texto y los encabezados
{0pt}                           % altura del pie de página
{2cm}                         % espacio entre el texto y el pie de página

\begin{document}

\begin{titlepage}

\begin{center}
\vspace*{-1in}
\begin{figure}[htb]
\begin{center}
\includegraphics[width=8cm]{udc.eps}
\end{center}
\end{figure}

\vspace*{1in}
PROGRAMACIÓN DE SISTEMAS 24/25 Q1\\
Icono de la aplicación\\
\vspace*{1in}
\begin{Large}
\textbf{Juego de air hockey con pantalla dividida} \\
\end{Large}

\vspace*{3in}

\begin{large}
\raggedleft
\textbf{Autores:}\\ José Manuel Fernández Montáns (j.m.fmontans@udc.es) \\
Mateo Rivela Santos (mateo.rivela@udc.es)\\
Hugo Mato Cancela (hugo.matoc@udc.es)\\
\textbf{Persona de contacto:} Hugo\\
\textbf{Fecha:}\textit{ A Coruña, 10 Octubre 2024}\\
\textbf{Versión:}\textit{0.1}\\
\textbf{Nombre de la aplicación:} \textit{Space Rend Hockey}\\
\end{large}

\end{center}
\end{titlepage} 

\newpage

\addtocontents{toc}{\hspace{-7.5mm} \textbf{Capítulos}}
\addtocontents{toc}{\hfill \textbf{Página} \par}
\addtocontents{toc}{\vspace{-2mm} \hspace{-7.5mm} \hrule \par}

\pagenumbering{empty}

\tableofcontents

\vspace{5cm}

\begin{flushright}
\begin{table}[hbtp]
\begin{center}

\caption{Tabla de versiones.}
\label{tabla:versiones}
\small
\vspace{1ex}

\begin{tabular}{|c|c|l|}
\hline
Versión & Fecha & Autor \\
\hline \hline
0.1 & 10/10/24 & Hugo\\ \hline
1.0 & 21/10/24 & Hugo\\ \hline

\end{tabular}

\end{center}
\end{table}
\end{flushright}


\newpage
\pagenumbering{arabic}


%%%%%%%
%%%%%%%
\section{Introducción}\label{cap.introduccion}

%%
\subsection{Objetivos}
El objetivo principal será trabajar con una pantalla dividida con una baja latencia y gestionar los rebotes para una pelota que viaja entre dispositivos y añadir un menú.

Podemos añadir notificaciones para que se una alguien a la partida, añadir un ranking, cambiar el tipo de pelota, añadir distintas dificultades iniciales, la pelota que aumente de velocidad con cada rebote de forma indefinida, ver las personas conectadas al juego, poner un estado a cada persona en base a si está en línea o no, o si ya está en una partida.
%%
\subsection{Motivación}
Intención de experimentar con traspaso de datos entre dispositivos e interacción de elementos de la aplicación con la pantalla del dispositivo.
%%
\subsection{Trabajo relacionado}
Existen aplicaciones similares, como es el caso de juegos de air hockey online. Nuestra aplicación busca conectar dos dispositivos de forma remota, a diferencia de las otras aplicaciones, que apuestan por una pantalla dividida para dos jugadores. Destacar que la base del juego es muy parecida, pero sin tener que conectar pantallas.

%%%%%%%
%%%%%%%
\section{Análisis de requisitos}

%%
\subsection{Funcionalidades}
Conectar dos teléfonos, cada uno mostraría una mitad del campo completo, de forma que los elementos en los que cada usuario se tiene que fijar pueden ser de mayor tamaño.
 
Los mazos con los que cada uno tiene que golpear el disco se controlarían con un solo dedo, apareciendo en la parte de la pantalla donde se mantiene este.
 
El disco, al ser golpeado por uno de los mazos, se mueve en una trayectoria determinada por el ángulo (y, preferiblemente, fuerza) del impacto, pudiendo ser alterada al chocar con los límites del campo, que corresponderían a los bordes del dispositivo.
%%
\subsection{Prioridades}
Lo principal sería lograr la conexión entre dispositivos, pasar los datos de forma correcta y apropiada, que el disco pase y rebote para implementar el juego de air hockey.
Primero, se implementará el trapaso de objetos entre pantallas y dispositivos con conexión mediante Bluetooth.
Luego, trabajaremos con rebotes y bordes de la pantalla como paredes.
Una vez realizado lo más básico, añadiremos una pantalla de menú para acceder a los ajustes y a prototipos anteriores. 
Por último, uniremos las funcionalidades básicas, añadiremos bloques destructibles que funcionarán como paredes.

%%%%%%%
%%%%%%%
\section{Planificación inicial}

%%
\subsection{Iteraciones}
Aqui se definen los prototipos de nuestro proyecto y las funcionalidades principales que trabajamos en cada prototipo. Se implementarán en el orden establecido:
\begin{description}
\item[P1:] Pasar un objeto entre dos dispositivos conectados por Bluetooth.
\item[P2:] Implementar los rebotes y los bordes de la pantalla del dispositivo como paredes.
\item[P3:] Añadimos una maza que el usuario puede manejar y que rebota con la pelota.
\item[P4:] Añadir un menú con el que acceder a la pantalla de ajustes y a los prototipos anteriores.
\item[P5:] Unión de funcionalidades de P1 y P2: un disco que rebota y se transporta entre dos pantallas al atravesar el borde superior.
\item[P6:] Añadir bloques destructibles que funcionen como paredes.
\item[P7:] Producto final
\end{description}
%%
\subsection{Responsabilidades}
\begin{description}

\item[Físicas:] Mateo 
\item[Conexión:] José Manuel
\item[Visualización:] Hugo
\end{description}

%%
\subsection{Hitos}
\begin{itemize}
\item Establecer una conexión de baja latencia entre ambos dispositivos.
\item Implementar rebotes.
\item Hacer que los bordes de la pantalla funcionen como paredes.
\item Añadir bloques destructibles que funcionen como paredes.
\item Crear un menú con varios ajustes.
\end{itemize}

Entregables:
Cada protipo será una entrega.

%%
\subsection{Incidencias}
No tenemos conocimientos sobre gráficos o físicas por lo que tendremos que aprender y probar todo lo relacionado a eso. Además, para paliar esas carencias, tenemos pensado usar LibGDX\cite{misc-libGDX} y KryoNet\cite{misc-kryonetl} para la codificación general de juego. Y usaremos las librerías Box2d\cite{misc-box2d}, Tween Engine\cite{misc-tween} y BluetooLib\cite{misc-bluetooth} para las físicas, las animaciones y la conexión Bluetooth respectivamente.
\par
En la primera iteración, probaremos la conexión entre dispositivos y que el objeto se pasa correctamente entre ellos.
En la segunda, en un prototipo distinto, el objeto debe rebotar contra los bordes de la pantalla de un dispositivo para que funcionen como paredes
En la tercera, implementamos la maza, que debe responder correctamente al input del usuario, y nos aseguramos que la pelota rebote correctamente contra esta.
En la cuarta, el menú nos debe permitir cambiar entre varios prototipos
En la quinta, uniremos las funcionalidades de las dos primeras iteraciones y nos aseguraremos de que funcionan correctamente ambas al mismo tiempo
En la sexta, añadiremos bloques que funcionen como paredes. La pelota debe rebotar contra estos como lo hace con los bordes de la pantalla.
En la última, comprobaremos que, tras haber implementado todo, funciona correctamente.
\par
En caso de que alguien enferme repartiremos el trabajo asignado a esa persona entre los demás integrantes del grupo.

%%%%%%%
%%%%%%%
\section{Diseño}
\subsection{Arquitectura}
Se hará uso de una arquitectura centralizada cliente servidor, en el cual existen dos componentes principales:
\begin{itemize}
    \item Cliente: Es el componente que solicita y consume servicios. Un cliente puede ser una aplicación o dispositivo que se conecta al servidor para enviar solicitudes (como pedir datos, ejecutar acciones, etc.). El cliente generalmente ofrece una interfaz de usuario y gestiona la interacción directa con este.
    \item Servidor: Es el componente que provee los servicios o recursos solicitados por el cliente. Un servidor puede manejar múltiples solicitudes de clientes simultáneamente y responde con los datos o resultados necesarios (como el procesamiento de datos, almacenamiento, etc.). El servidor puede estar ubicado en una red local o en la nube.
\end{itemize}


\subsection{Persistencia}
Respecto a qué información almacenaremos, ambos jugadores siempre tendrán los datos de la posición, ángulo y fuerza de la pelota. Además, cada uno guardará dónde está situada su propia maza en caso de que esté siendo usada.
\subsection{Vista}
En lo que respecta al diseño de la aplicación, tendremos una única actividad a priori en la que, mediante fragmentos, iremos añadiendo la dinámica. El manejo del juego en sí se hará mediante un view personalizado , mostrando todos los elementos y actualizando sus respectivas posiciones cada cierto tiempo. 

\subsection{Comunicaciones}
La comunicación en este juego es una parte fundamental. Para ello, buscamos un sistema de baja latencia. Para esta conexión, se contará con un servidor central al que cada cliente se conectará por internet; de esta forma, se gana en rango de uso (rango global). Existe un claro problema y es que el servidor del que se va a hacer uso es gratuito y, por lo tanto, la latencia va a ser bastante alta, pero hemos llegado a la conclusión de que es irrelevante para este proyecto porque, como cada usuario en una partida solo va a ver su pantalla a priori, la latencia es totalmente irrelevante.

\subsection{Sensores}
Se hace uso del sensor táctil, con este se implementará el movimiento del mazo en cuestión. Es un dispositivo que detecta y responde al contacto físico o la presión sobre su superficie, en este caso no se hará uso de la detección de la presión. Para la implementación del rebote se puede tener o no en cuenta la velocidad con la que se desplaza en contacto con el panel táctil pero no es algo que vayamos a implementar en nuestro caso desde a priori.
\subsection{Trabajo en background}
Se enviarán los datos de la posición y ángulo de la pelota, cada X tiempo..Para controlar la velocidad de actualización de la pantalla y evitar sobrecargar la CPU, usaremos un Thread para controlar el tiempo entre re-dibujos.
\\

%%%%%%%

\section{Bibliografía}
\bibliographystyle{pfc-fic}
\bibliography{biblio}


\end{document}